\chapter{Технологический раздел}

\section{Выбор языка программирования}
В качестве языка программирования был выбрал язык \textbf{Java}. Данный язык активно используется в качестве написания серверов веб приложений, а также имеет большое количество библиотек и фреймворков, облегчающих создание целостного веб приложения.

При создания приложения использовалась среда разработки программного обеспечения \textbf{IntelliJ IDEA} от компании JetBrains. Данная IDE очень удобна для написания и отладки программ, имеет большое количество горячих клавиш, ускоряющих написание кода.  

\section{Модель}

Модель представлена Java - классом.

\lstinputlisting[language=Java, caption=Пример модели Book.]{code/Book.java} 

Класс модели с помощью ORM отображаются на реляционную таблицу базы данных. Концепцию ORM в Java реализует JPA. \textbf{Java Persistence API (JPA)} — API, входящий с версии Java 5 в состав платформ Java SE и Java EE, предоставляет возможность сохранять в удобном виде Java-объекты в базе данных.

\lstinputlisting[language=Java, caption=Пример модели Book в контексе ORM.]{code/BookJPA.java} 

\section{Вид}

Вид создан с помощью JSP. \textbf{JSP (JavaServer Pages)} — технология, позволяющая веб-разработчикам создавать содержимое, которое имеет как статические, так и динамические компоненты. Страница JSP содержит текст двух типов: статические исходные данные, которые могут быть оформлены в одном из текстовых форматов HTML, SVG, WML, или XML, и JSP- элементы, которые конструируют динамическое содержимое.

\lstinputlisting[language=Html, caption=Пример страницы list-tasks.jsp]{code/list-tasks.jsp} 


\section{Контроллеры и веб сервер}

Контроллеры реализуются Java сервлетами. \textbf{Сервлет} является интерфейсом Java, реализация которого расширяет функциональные возможности сервера. Сервлет взаимодействует с клиентами посредством принципа запрос-ответ. Хотя сервлеты могут обслуживать любые запросы, они обычно используются для расширения веб-серверов.

В качестве веб сервера использутся Apache Tomcat. \textbf{Tomcat} — контейнер сервлетов с открытым исходным кодом, разрабатываемый Apache Software Foundation. Реализует спецификацию сервлетов и спецификацию JavaServer Pages (JSP). Написан на языке Java.

\lstinputlisting[language=Java, caption=Пример сервлета CreateTaskServlet]{code/CreateTaskServlet.java}

\section{Навигация по сайту} 

\subsection{Для неавторизованного пользователя} 

\begin{figure}[h]
  \centering
  \includegraphics[width=\textwidth]{hatnotauth.png}
  \caption{ Шапка сайта для неавторизованного пользователя.}
\end{figure}

Нажав на соответствующие ссылки можно перейти к страницам регистрации и авторизации.

\subsection{Для авторизованного пользователя} 

\begin{figure}[h]
  \centering
  \includegraphics[width=\textwidth]{hatauth.png}
  \caption{ Шапка сайта для авторизованного пользователя.}
\end{figure}

Перейдя по соответствующим ссылкам, можно:
\begin{enumerate}
\item редактировать профиль
\item выйти
\item создать задачу
\item добавить книгу
\item перейти к списку своих задач
\item осуществить поиск по задачам
\item выбрать категорю для отображения задач/книг
\item перейти к списку активных задач
\item перейти к списку книг
\end{enumerate}