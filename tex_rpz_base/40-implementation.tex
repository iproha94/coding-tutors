\chapter{Технологический раздел}
\section{Используемые технологии}
\subsection*{Язык программирования}
Поставленная задача требует много работы с I/O: загрузка файлов и работа с базой данных. Вычислений производится относительно немного (стемминг и выделение основного содержимого), поэтому очевиден выбор асинхронной модели. Относительно низкоуровневые варианты (Си + libuv/libev/libevent или rust + mio) не дадут выигрыша из-за малого количества вычислений. Первоначально синхронные python и lua имеют синхронные обёртки баз данных, что усложняет их встраивание в существующие асинхронные фреймворки (в случае с python это asyncio). С другой стороны, есть первоначально асинхронный и при этом достаточно популярный сегодня node.js.

Дополнительным преимуществом использования node.js является единый язык программирования (javascript) на клиенте и сервере. Таким образом, в качестве ЯП был выбран EcmaScript 2015~--- будущая версия ЯП JavaScript. Однако его поддерживают  пока далеко не все браузеры, поэтому исходный код для поисковой страницы транслируется компилятором babel в Javascript 1.5~--- предыдущую версию языка.


\subsection*{База данных}
В качество базы данных было решено взять небольшую, но при этом достаточно функциональную и быструю, СУБД SQLite, которая размещает базу данных в одном единственном файле (не считая временных файлов журнала). Однако запросы специально составлялись наиболее переносимым образом, чтобы программу можно было легко адаптировать к другим СУБД (например, PostgreSQL). Главным преимуществом использования встраиваемой базы данных является просто установки и настройки: нет необходимость запускать сервер СУБД, посколько логика его работы <<встраивается>> в приложение.


\subsection*{Используемые библиотеки}
В среде node.js-разработки общепринят UNIX-подход: множество небольших библиотек, каждая из которых решает только одну задачу, но эффективно. Стандартный пакетный менеджер позволяет легко устанавливать все зависимости одной командой.

\begin{description}
  \item[bloomfilter] эффективная реализация фильтра Блума;
  \item[sqlite3] асинхронный интерфейс к sqlite;
  \item[entities] обнаружение и замена мнемоник (X)HTML;
  \item[htmlparser2] высокопроизводительный SAX-парсер (X)HTML;
  \item[koa] небольшой асинхронный веб-фреймворк;
  \item[natural] работа с естественными языками;
  \item[priorityqueuejs] высокопроизводительная реализация пирамиды;
  \item[readabilitySAX] выделение основного содержимого;
  \item[request] упрощение запросов;
  \item[yargs] построение командных интерфейсов;
\end{description}


\subsection*{Окружение разработчика}
В качестве редактора кода был использован Vim на ОС Arch Linux. Сборка проекта осуществляется пакетным менеджером npm~--- родным инструментом для разработчиков на JS. В качестве системы контроля версий использовался git.


\section{Пользовательский интерфейс}
\subsection*{Интерфейс командной строки}
